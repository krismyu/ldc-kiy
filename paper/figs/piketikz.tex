% Data flow diagram
% Author: David Fokkema
\documentclass{article}
\usepackage{tikz}
\usetikzlibrary{arrows,fit,calc,decorations.markings}
\tikzstyle{container} = [draw, very thick, dashed, rectangle, rounded
corners, inner sep=2.5em, fill = yellow!20 ]

\begin{document}

% Define the layers to draw the diagram
\pgfdeclarelayer{background}
\pgfdeclarelayer{foreground}
\pgfsetlayers{background,main,foreground}

\begin{center}
\begin{tikzpicture}[
  font=\sffamily,
  every matrix/.style={ampersand replacement=\&,column sep=5cm,row sep=2cm},
  source/.style={draw,thick,rounded corners,fill=blue!20,inner sep=.2cm},
  sink/.style={source,fill=green!20},
  dots/.style={gray,scale=2},
  to/.style={->,>=stealth',shorten >=1pt,semithick,font=\sffamily\footnotesize},
  every node/.style={align=center},
    decoration={
      markings,
      mark=at position 1 with {\arrow[scale=1,black]{triangle 60}};
    }
]


  % Position the nodes using a matrix layout
  \matrix{
    % Row 1
%    \node[sink] (words) {\textbf{UNCLASSIFIED}\\ \textbf{WORDS}};  \& \& \\
    % Row 2
      \node[source] (morph) {\textbf{Classify word (morphology)} \\
      Lexical class (N, V, Adj, $\ldots$) \\ Morphological complexity
      \\ $\vdots$}; 

       \& \node[source] (sub-frame) {\textbf{Choose substitution
          frame} \\ Pragmatic context \\ Prosodic position for word \\
      $\vdots$};\\


    % Row 3
    \node[source] (phon) {\textbf{Classify word (phonology)}\\ Length (TBU) \\
      Syllabic structure \\ Stress assignment \\ Segmental structure
      \\ $\vdots$};

       \&  \node[source] (tone) {\textbf{Classify words by pitch
          contour}};\\
      \&\&\\ % Buffer row
      % Row 4
%    \node[sink] (classes) {\textbf{WORD CLASSES} \\ \textbf{(Substitution lists)}}; \& \& \\
    };



    \begin{pgfonlayer}{background}

 \node[container, fit=(phon) (morph)] (class) {};
    \node at (class.north) [below, node distance=0 and 0]
    {\textbf{Classify words by} \\ \textbf{morpho-phonological structure}};

 \node[container, fit=(sub-frame) (tone)] (sub) {};
    \node at (sub.north) [below, node distance=0 and 0]
    {\textbf{Classify substitution items by pitch} \\ \textbf{in controlled contexts}};

  % % Since these boxes are nodes, it is easy to put text above or below them                          
  % \node at (first dotted box.north) [above, inner sep=3mm] {\textbf{Transmitter}};
  % \node at (second dotted box.south) [below, inner sep=3mm] {\textbf{Receiver}};
    \end{pgfonlayer}

    % Relative node placement
    \node[sink] at (class.west) {\textbf{UNCLASSIFIED} \\ \textbf{WORDS}}; 
    \node[sink] at (class.east) {\textbf{WORD CLASSES} \\ \textbf{(Substitution lists)}}; 

  % Draw the arrows between the nodes and label them.

 % 1st pass: draw arrows
%\draw[line width=1mm, double, double distance=1mm] (class) -- (sub);
%\draw[
%    -triangle 45,
%    line width=1.0mm,
%    postaction={draw=green, line width=0.25cm, shorten >=0.0cm, -}
%] (class) -- (sub);
%\tikz\draw[line width=1mm,-implies,double, double distance=1mm] (0,0)
%-- (1,0);
%    \draw [line width=2.0mm, postaction={decorate}, shorten >=10pt] (class) -- (sub);
  % \draw[to] (morph) -- node[midway,right] {raw event data\\level 1} (morph);
  % \draw[to] (morph) --
  %     node[midway,right] {raw event data\\level 1} (monitor);
  \draw[to] (sub-frame) -> (tone);
  % \draw[to] (sub-frame) to[bend right=50] node[midway,above] {tone}
  %     node[midway,below] {level 1} (tone);
  % \draw[to] (tone) to[bend right=50] node[midway,above] {sub-frame}
  %     node[midway,below] {level 1} (sub-frame);
  % \draw[to] (monitor) -- node[midway,above] {events}
  %     node[midway,below] {level 1} (datastore);

\end{tikzpicture}
\end{center}
\end{document}
